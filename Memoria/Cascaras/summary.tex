%---------------------------------------------------------------------
%
%                      summary.tex
%
%---------------------------------------------------------------------
%
% Contiene el cap�tulo del resumen.
%
% Se crea como un cap�tulo sin numeraci�n.
%
%---------------------------------------------------------------------

\chapter{Summary}
\cabeceraEspecial{Summary}

\begin{FraseCelebre}
\begin{Frase}

\end{Frase}
\begin{Fuente}

\end{Fuente}
\end{FraseCelebre}

By law, the educational curriculum seeks to ensure that the training process of students has the necessary materials, resources, and content during their academic journey, under equal conditions. However, since not all students have the same learning tools, it is possible to make curricular adaptations.

Currently, there are no free tools to make curricular adaptations, and teachers spend too much time creating academic material for students who need it: they have to adjust the text font, size, search for images on the Internet or scan them from books, etc., and some even have to do it by hand.

This project consists of the development of a web application, AdaptaMaterialEscolar, created specifically to facilitate teachers the non-significant curricular adaptation of exams, activities, and personalized teaching units for each student. For this purpose, we have implemented a text editor that has the formats and functionalities most used by teachers, so that they can make the adaptations more quickly, following a User-Centered Design. The functionalities that have been implemented in AdaptaMaterialEscolar are a pictogram finder and exercises to fill in the blanks, generate word searches, true or false, define concepts, and questions to develop with limited space.

The web application has been tested and evaluated by end-users who intend to use it in the school period. The results obtained show that a real need for teachers and their interest in continuing to develop and improve AdaptaMaterialEscolar is covered.

\endinput
% Variable local para emacs, para  que encuentre el fichero maestro de
% compilaci�n y funcionen mejor algunas teclas r�pidas de AucTeX
%%%
%%% Local Variables:
%%% mode: latex
%%% TeX-master: "../Tesis.tex"
%%% End:
